\documentclass[a4paper,12pt]{article} % добавить leqno в [] для нумерации слева
\usepackage[a4paper,top=1.3cm,bottom=2cm,left=1.5cm,right=1.5cm,marginparwidth=0.75cm]{geometry}
%%% Работа с русским языком
\usepackage{cmap}					% поиск в PDF
\usepackage{mathtext} 				% русские буквы в фомулах
\usepackage[T2A]{fontenc}			% кодировка
\usepackage[utf8]{inputenc}			% кодировка исходного текста
\usepackage[english,russian]{babel}	% локализация и переносы

\usepackage{siunitx} % значок градуса

\usepackage{enumitem} % крутые списки

\usepackage{relsize} % увеличение формул с помощью \mathlarger{}

\usepackage{graphicx} % вставка фотографий
\graphicspath{ {./images} } % путь к папке с фотографиями
\usepackage{wrapfig} % текст вместе с фотографией

\usepackage{tikz} % рисунки
\usetikzlibrary{shapes.geometric} % для кривых множественных порядков

\usepackage{pst-math,pst-plot} % ещё один графический редактор

\usepackage{pgfplots} % построение графиков внутри латеха
\usepgfplotslibrary{external}
\tikzexternalize % ускорение работы при загрузке графиков

\usepackage{multirow} % объединенные ячейки в таблицах
\usepackage{multicol} % текст разделенный на неколько колонн

\usepackage{ragged2e} % выравневание текста

\usepackage{wrapfig}
\usepackage{tabularx} % более гибкие таблицы

%%% Дополнительная работа с математикой
\usepackage{amsmath,amsfonts,amssymb,amsthm,mathtools} % AMS

%% Шрифты
\usepackage{dsfont}
\usepackage{euscript}	 % Шрифт Евклид
\usepackage{mathrsfs} % Красивый матшрифт
\usepackage{bbold}

\begin{document}

\begin{titlepage}
    \begin{center}
        \vspace*{1cm}
            
        \Huge
        \textbf{Линейная Алгебра}
            
        \vspace{0.5cm}
        \LARGE
        Второй семестр\\
        Первое задание
            
        \vspace{1.5cm}
            
        \textbf{Бородулин Егор \\ Б02-204}
        
        \vfill
        
        \includegraphics[scale = 0.55]{images/DGAP.jpg}
        
        \vspace{0.8cm}
            
        \Large
        ФОПФ\\
        МФТИ\\
        Долгопрудный\\
        8.02.2023
            
    \end{center}
\end{titlepage}

\section*{Б20.3}$\ $

3,4) Решим задачу в общем виде:\\
Пусть для $\pmb\xi=(\xi_1,\cdots,\xi_n)^{T}\in\mathcal{L}_n$ выполняется $$\sum_{\alpha=1}^{n}\xi_{\alpha}=\lambda\in\mathbb{R}$$
Возьмем два произольных вектора $\pmb v$ и $\pmb u$, для которых выполняется данное условие, тогда их сумма будет записываться так:
$$\pmb{v}+\pmb{u}=(v_1+u_1,\cdots,v_n+u_n)^{T}$$
Тогда:
$$\sum_{\alpha=1}^{n}(v+u)_{\alpha}=\sum_{\alpha=1}^{n}v_{\alpha}+\sum_{\alpha=1}^{n}u_{\alpha}=2\lambda\in\mathbb{K}$$
Таким образом сумма векторов, удовлетворяющих заданному условию, удовлетворяет данному условию только при $\lambda=0$, заметим, что в этом случае $k\pmb{\xi}$ будет также удовлетворять данному условию, так что только при $\lambda=0$ данное условие будет линейным, а векторы удовлетворяющие ему будут образовывать подпространство в $\mathcal{L}_n$.

Базисом в таком подпространстве можно выбрать векторы вида $e_i=(0,\cdots,0,1,-1,0,\cdots,0)$, где i-тая координата равна 1 (такая система векторов будет действительно линейно независимой, так как в разложении любого вектора из данной системы не могут участвовать "крайние" векторы). Так как базисных векторов $n-1$, то размерность данного подпространства равна $n-1$
\section*{Б20.4}

2)Пусть два вектора перпендикулярны данной прямой, тогда через них проходит единственная плоскость. Так как оба вектора перпендикулярны данной прямой, то построенная плоскость будет перпендикулярна данной прямой. Сумма любых векторов из данной плоскости и произведение таких на скаляр будет оставаться лежать в данной плоскости, так что они будут образовывать линейное подпространство размерности 2.
\section*{Б20.6}\label{Б20.6}

2,3) Из свойств сложения матриц и умножения их на элемент из $\mathbb{K}$ следует, что верхнетреугольные матрицы образуют подпространство в пространстве $M_{n\times n}$. Для верхнетреугольных матриц можно выбрать базис из матричных единиц, являющихся также верхнетреугольными матрицами, которых будет $\frac{n(n+1)}{2}$. Так как диагональные матрицы будут частным случаем верхнетреугольных матриц, то они будут тоже образовывать подпространство размерности $n$.

5)Аналогично верхнетреугольным матрицам кососимметричные матрицы образуют подпространство в $M_{n\times n}$, их базисом будут матрицы вида $E_{ij}-E_{ji}$ - всего таких будет $\frac{n(n-1)}{2}$
\section*{Б20.8}3)
Так как сумма нечетных многочленов тоже нечетный многочлен, то они образуют подпространство в $\mathcal{P}_n$ с базисными векторами вида $x^{2k+1}$, которых будет $\begin{bmatrix}\frac{n+1}{2}\end{bmatrix}$
\section*{Б20.14}9)
$$\begin{pmatrix*}
    1&1&1&1\\
    2&2&2&2\\
    3&3&3&3\\
    1&1&2&2
\end{pmatrix*}\rightarrow
\begin{pmatrix*}
    1&1&1&1\\
    0&0&0&0\\
    0&0&0&0\\
    1&1&2&2
\end{pmatrix*}$$
Базис - $\begin{Bmatrix}(1,1,1,1)^{T},&(1,1,2,2)^{T}\end{Bmatrix}$, откуда размерность равна 2.
\section*{Б20.18}
Переход от одной системы векторов к другой осуществляется посредством матрицы линейного оператора. Так как векторов в столько же, сколько в базисе матриц порядка 2, то требуется невырожденность данной матрицы.
Выпишем элементы матричных единиц в стобцы, получив единичную матрицу:
$$\begin{pmatrix*}
    1&2&1&3\\
    -1&5&1&4\\
    1&1&0&5\\
    -1&3&1&7
\end{pmatrix*}=
\begin{pmatrix*}
    1&0&0&0\\
    0&1&0&0\\
    0&0&1&0\\
    0&0&0&1
\end{pmatrix*}A$$
Заметим, что для невырожденности A достаточно равенства её ранга её порядку.
$$\begin{pmatrix*}[r]
    1&2&1&3\\
    -1&5&1&4\\
    1&1&0&5\\
    -1&3&1&7
\end{pmatrix*}\rightarrow
\begin{pmatrix*}[r]
    1&2&1&3\\
    0&7&2&7\\
    0&-1&-1&2\\
    0&5&2&0
\end{pmatrix*}\rightarrow
\begin{pmatrix*}[r]
    1&0&-1&7\\
    0&0&-5&21\\
    0&1&1&-2\\
    0&0&-3&10
\end{pmatrix*}
\rightarrow
\begin{pmatrix*}[r]
    1&0&-1&7\\
    0&0&1&1\\
    0&1&1&-2\\
    0&0&-3&10
\end{pmatrix*}
\rightarrow
\begin{pmatrix*}[r]
    1&0&0&8\\
    0&0&1&1\\
    0&1&0&-3\\
    0&0&0&13
\end{pmatrix*}$$
Поэтому данные матрицы, являясь системой векторов, образуют базис в пространстве матриц порядка 2. Используя закон преобразования векторов получаем выражение для $A_{26}$
$$A^{'}_{26}=A^{-1}A_{26}=\begin{pmatrix*}[r]
    \frac{13}{37}&\frac{1}{37}&\frac{11}{37}&-\frac{14}{37}\\
    -\frac{3}{37}&\frac{14}{37}&\frac{6}{37}&-\frac{11}{37}\\
    \frac{36}{37}&-\frac{20}{37}&-\frac{35}{37}&\frac{21}{37}\\
    -\frac{2}{37}&\frac{3}{37}&\frac{4}{37}&\frac{5}{37}
\end{pmatrix*}\begin{pmatrix*}
    5\\
    14\\
    6\\
    13
\end{pmatrix*}=(-1,2,-1,1)^{T}$$
\section*{Б20.20}
Заметим, что замена $t_1=t-\alpha$ обратима, $t=t_1+
\alpha$, так что многочлены вида $(t-
\alpha)^k$ образуют базис в пространстве многочленов.
Считая координатами многочлена - последовательность коэффициентов, то матрицей перехода к новому базису будет:
$$((t-\alpha)^{n},\ (t-\alpha)^{n-1},\ \cdots,\ t-\alpha, 1)=
(t^{n},\ t^{n-1},\ \cdots,\ t, 1)
\begin{pmatrix*}
    1&0&\cdots&0&0\\
    C^{n}_{2}&1&\cdots&0&0\\
    \vdots&\vdots&\ddots&\vdots&\vdots\\
    n&n-1&\cdots&1&0\\
    1&1&\cdots&1&1
\end{pmatrix*}
$$
Так что:
$$p(t-\alpha)=
\begin{pmatrix*}
    1&0&\cdots&0&0\\
    C^{n}_{2}&1&\cdots&0&0\\
    \vdots&\vdots&\ddots&\vdots&\vdots\\
    n&n-1&\cdots&1&0\\
    1&1&\cdots&1&1
\end{pmatrix*}^{-1}p(t)
$$

\section*{Б20.22*}
3)$$\begin{pmatrix*}
-3&1&-2\\
6&-2&4\\
-15&5&-10
\end{pmatrix*}x=\textbf{o}\Leftrightarrow\begin{pmatrix*}
-3&1&-2\\
0&0&0\\
0&0&0
\end{pmatrix*}x=\textbf{o}$$$$a=\alpha\begin{pmatrix*}
-\frac{1}{3}\\
1\\
0
\end{pmatrix*}\qquad\qquad\qquad\qquad\qquad\qquad b=\beta\begin{pmatrix*}
\frac{2}{3}\\
0\\
1
\end{pmatrix*}$$
Так, мы получили уравнение плоскости $-3x+y-2y=0$ в пространстве $\mathbb{R}^3$, базис полученного двумерного пространства можно получить взяв, например, $\alpha=\beta=3$
$$\begin{pmatrix*}
-1\\
3\\
0
\end{pmatrix*}\qquad\qquad\qquad\qquad\qquad\qquad\begin{pmatrix*}
2\\
0\\
3
\end{pmatrix*}$$
\section*{Б20.23}4)
Столбцы $c_{166}$ и $c_{196}$ составляют стоблцы фундаментальной матрицы.
$$x=\Phi h + x_0\Rightarrow Ax=A(\Phi h + x_0)=b\Rightarrow A\Phi h=o\ (\forall h)\Rightarrow A\Phi=O\Rightarrow A\Phi B_{n-r}=O$$
Под $B_{n-r}$ имеется в виду такая невырожденная (для обратимости) квадратная матрица порядка $n-r$, что $\Phi B_{n-r}$ равна нормальной фундаментальной матрице данной системы, хотя общий вид матриц $A$ задается также решением системы уравнений:
\begin{equation}
    \begin{pmatrix*}
a_{11}&a_{12}&a_{13}&a_{14}\\
a_{21}&a_{22}&a_{23}&a_{24}\\
\end{pmatrix*}\begin{pmatrix*}
1&1\\
1&2\\
1&1\\
1&3
\end{pmatrix*}=\begin{pmatrix*}
0&0\\
0&0
\end{pmatrix*}
\label{eqn:SOLE1}
\end{equation}
В любом случае, видно, что данные уравнения задают систему уравнений с более чем 4-мя уравнениями с удвоенным количеством переменных, отчего однозначный ответ получить не представляется возможным.\\
Также можно заметить, что любую матрицу $A$ можно получить учитывая, что она сводится к упрощенной матрице размеров $r\times n$. Так как фундаметнальная матрица имеет размеры $n\times(n-r)$, то по её размерам можно восстановить размеры и вид упрощенной матрицы (подразумеваю матрицу вида \eqref{eqn:SOLE1}). Остальные матрицы коэффициентов и соответствующие им системы, задающие нужное данную фундаментальную матрицу, получаются, как линейные комбинации строк полученной упрощенной матрицы.\\
Транспонируем уравнение \eqref{eqn:SOLE1}:
\begin{equation*}\left(
\begin{pmatrix*}
a_{11}&a_{12}&a_{13}&a_{14}\\
a_{21}&a_{22}&a_{23}&a_{24}\\
\end{pmatrix*}\begin{pmatrix*}
1&1\\
1&2\\
1&1\\
1&3
\end{pmatrix*}\right)^{T}=\begin{pmatrix*}
1&1\\
1&2\\
1&1\\
1&3
\end{pmatrix*}^{T}
    \begin{pmatrix*}
a_{11}&a_{12}&a_{13}&a_{14}\\
a_{21}&a_{22}&a_{23}&a_{24}\\
\end{pmatrix*}^{T}=
\end{equation*}
\begin{equation}
=\begin{pmatrix*}
1&1&1&1\\
1&2&1&3
\end{pmatrix*}\begin{pmatrix*}
a_{11}&a_{21}\\
a_{12}&a_{22}\\
a_{13}&a_{23}\\
a_{14}&a_{24}
\end{pmatrix*}
=\begin{pmatrix*}
0&0\\
0&0
\end{pmatrix*}
\label{eqn:SOLE2}
\end{equation}
Иначе говоря, транспонированная нужная матрица коэффициентов является фундаментальной для транспонированной данной фундаментальной матрицы.\\
Методом Гаусса получаем:
\begin{equation}
\begin{pmatrix*}
1&1&1&1\\
1&2&1&3
\end{pmatrix*}\rightarrow\begin{pmatrix*}
1&1&1&1\\
0&1&0&2
\end{pmatrix*}\rightarrow\begin{pmatrix*}
1&0&1&-1\\
0&1&0&2
\end{pmatrix*}\Rightarrow A^{T}=\begin{pmatrix*}[r]
-1&1\\
0&-2\\
1&0\\
0&1
\end{pmatrix*}\Rightarrow A=\begin{pmatrix*}[r]
-1&0&1&0\\
1&-2&0&1
\end{pmatrix*}
\label{eqn:SOLE3}
\end{equation}
Опять же, ответ не однозначен.\\
Задачу можно было решить нахождением обратной матрицы к квадратной матрице снизу фундаментальной:
$$\begin{pmatrix*}
1&1\\
1&2\\
1&1\\
1&3
\end{pmatrix*}{\begin{pmatrix*}[r]
\frac{3}{2}&-\frac{1}{2}\\
-\frac{1}{2}&\frac{1}{2}
\end{pmatrix*}}=\begin{pmatrix*}
1&0\\
\frac{1}{2}&\frac{1}{2}\\
1&0\\
0&1
\end{pmatrix*}\Rightarrow A=\begin{pmatrix*}[r]
1&0&1&0\\
0&1&-\frac{1}{2}&-\frac{1}{2}
\end{pmatrix*}$$
Домножением полученной матрицы на $\begin{pmatrix*}[r]
-1&0\\
1&-2\\
\end{pmatrix*}$
получаем матрицу из \eqref{eqn:SOLE3}. \\
В каком-то смысле другие матрицы будут матрицей из \eqref{eqn:SOLE3}, записанной в других базисах.
\section*{Б20.29}
а,б) Так как матрица перехода является сторокой, элементами которой являются вектор-столбцы нового базиса в старом, то  при перестановке i-го и j-го базисного вектора в новой системе, то столбцы соответствующих номеров поменяются местами. При перестановке i-го и j-го базисного вектора в старой системе, в матрице перехода переставятся строки соответствующих номеров.
в)При расположении базисных векторов обеих систем в обратном порядке получается транспонированная матрица.
\section*{T.1*}
Заметим, что множества с такой операцией образуют группу. Подгруппой минимального порядка будет группа из одно множества и пустого, так что следующая подгруппа минимального порядка, большего данной будет группа порядка 4. Таким образом строится флаг подгрупп, максимальной групой будет данное множество подмножеств. Также заметим, что групп простого порядка, кроме как порядка 2 быть не может, так как для каждой пары множест входит также из симметрическая разность. По теореме Лагранжа получаем $\#2^\mathcal{M}=2^n$
\section*{T.2*}
Так как векторное пространство стоится над полем, то аддитивная группа того поля должна быть циклической, а также $\#\mathbb{Z}=\#\mathbb{V}$. Рассмотрим изоморфизм $\varphi:\mathbb{K}\to\mathbb{Z}$.\\
Пусть $\varphi(k)=1$, $k$ - порождающий элемент $(\mathbb{K},+)$, однако из аксиом поля следует, что $\exists \frac{k}{2}$, из-за чего $2\varphi\left(\frac{k}{2}\right)=1\Rightarrow\varphi\left(\frac{k}{2}\right)\not\in\mathbb{Z}\Rightarrow \varphi \not\in\mathbf{Hom}((\mathbb{Z},+),(\mathbb{K},+))$ - таким образом определить порождающий элемент аддитивной группы поля счетной мощности не представляется возможным. 
\section*{Б21.2}
$$P^{2k}(x)=P^{2m+1}(x)=-P^{2m+1}(-x)=-P^{2k}(-x)=-P^{2k}(x)\Rightarrow P^{2k}(x)=0$$
$$P^{2k}\cap P^{2k+1}={0}$$
$$P^{n}(x)=\frac{P^{n}(x)+P^{n}(-x)}{2}+\frac{P^{n}(x)-P^{n}(-x)}{2}$$
$$\frac{P^{n}(x)+P^{n}(-x)}{2}\in P^{2k}$$
$$\frac{P^{n}(x)-P^{n}(-x)}{2}\in P^{2k+1}$$
$$P^{2k}+ P^{2k+1}=P^n$$
$$P^{2k}\oplus P^{2k+1}=P^{n}$$
\section*{Б21.3*}2)
$$rk(A)=rk(A^{T})$$
$$rk(A^{T}x)=n-rk(A^{T})=n-rk(A)$$
$$a\in A:A^{T}a=\textbf{o}\Leftrightarrow (aa)=0\Leftrightarrow a=\textbf{o}$$
\section*{Б21.6}5)
Составляя матрицу перехода к базису $\begin{Bmatrix}
a_1,&a_2,&b_1,&b_2
\end{Bmatrix}$:
\begin{equation*}
x'=\alpha_1a_1+\alpha_2a_2+\beta_1b_1+\beta_2b_2=S^{-1}x=
\begin{pmatrix*}[r]
\frac{29}{16}&\frac{3}{16}&-\frac{1}{2}&\frac{1}{2}\\
\frac{1}{8}&-\frac{1}{8}&0&0\\
-\frac{3}{2}&\frac{1}{2}&1&0\\
\frac{5}{16}&-\frac{5}{16}&-\frac{1}{2}&\frac{1}{2}
\end{pmatrix*}\begin{pmatrix*}[r]
1\\
-7\\
5\\
-2
\end{pmatrix*}=\begin{pmatrix*}[r]
-1\\
1\\
0\\
-1
\end{pmatrix*}
\end{equation*}
$$x_P=\alpha_1a_1+\alpha_2a_2=-a_1+a_2$$
\section*{Б21.7}
5) Векторы $a_1, a_2,a_3$ линейно независимы, так что они порождают арифметическое трехмерное пространство, их можно выбрать, как базис $\langle a_1, a_2, a_3\rangle + \langle b_1, b_2\rangle$. Аналогично можно выбрать векторы из второй системы, как базис пересечения. Размерности получившихся пространств будут соответственно 3 и 2.\\ 
7) Составим матрицу $\left(a_1,a_2,a_3\ \vline\ b_1,b_2,b_3\right)$:
$$
\begin{pmatrix*}[r]
1&-1&0&\vline&1&3&4\\
2&8&10&\vline&4&-2&2\\
1&-6&-5&\vline&-1&6&5\\
3&5&8&\vline&5&3&8
\end{pmatrix*}$$
Заметим, что $a_1+a_2=a_3$ и $b_1+b_2=b_3$, так что исключим их из матрицы, как $o$.
$$
\begin{pmatrix*}[r]
1&0&\vline&1&4\\
2&10&\vline&4&2\\
1&-5&\vline&-1&5\\
3&8&\vline&5&8
\end{pmatrix*}\rightarrow\begin{pmatrix*}[r]
1&0&\vline&1&4\\
0&5&\vline&1&-3\\
0&-5&\vline&-2&1\\
0&8&\vline&2&-4
\end{pmatrix*}\rightarrow\begin{pmatrix*}[r]
1&0&\vline&1&4\\
0&5&\vline&1&-3\\
0&0&\vline&-1&-2\\
0&4&\vline&1&-2
\end{pmatrix*}\rightarrow
$$
$$
\rightarrow\begin{pmatrix*}[r]
1&0&\vline&0&2\\
0&5&\vline&0&-5\\
0&0&\vline&-1&-2\\
0&4&\vline&0&-4
\end{pmatrix*}\rightarrow
\begin{pmatrix*}[r]
1&0&\vline&0&2\\
0&1&\vline&0&-1\\
0&0&\vline&-1&-2\\
0&1&\vline&0&-1
\end{pmatrix*}\rightarrow\begin{pmatrix*}[r]
1&0&\vline&0&2\\
0&1&\vline&0&-1\\
0&0&\vline&1&2\\
0&0&\vline&0&0
\end{pmatrix*}
$$
За базис суммы можно взять векторы $a_1,a_2,b_1$ так что размерность суммы равна 3.\\
\section*{Б21.11}
Из наличия хотя бы одного вектора, разлогаемого однозначно, следует $\mathcal{P}\cap\mathcal{Q}=o$, иначе, прибавляя нетривиальную линейную комбинацию $o$ к разложению данного вектора, получилось бы отличное от данного разложение.\\
Так как $\mathcal{P}+\mathcal{Q}=\mathcal{L}$, по формуле Грассмана получаем $\mathrm{dim}\left(\mathcal{P}+\mathcal{Q}\right)=\mathrm{dim}\mathcal{P}+\mathrm{dim}\mathcal{Q}$\\
$$\blacksquare$$
\section*{Б21.12}
1) Из формулы Грассмана следует:\\
$$\mathrm{dim}\left(\mathcal{P}\cap\mathcal{Q}\right)=\mathrm{dim}\mathcal{P}+\mathrm{dim}\mathcal{Q}-\mathrm{dim}\left(\mathcal{P}+\mathcal{Q}\right)>0$$
2) $$\mathrm{dim}\left(\mathcal{P}\cap\mathcal{Q}\right)=m$$
$$\mathrm{dim}\left(\mathcal{P}+\mathcal{Q}\right)=n=m+1$$
$$m\leq\mathrm{dim}\mathcal{Q}\leq m+1$$
$$m\leq\mathrm{dim}\mathcal{P}\leq m+1$$
Так как $\mathrm{dim}:\mathcal{L}\to\mathbb{Z}^{+}$, то:
$$\mathrm{dim}\mathcal{P}, \mathrm{dim}\mathcal{Q}=\left[\begin{matrix}
    m,\\
    m+1;
\end{matrix}\right.$$
Так как $\mathrm{dim}\left(\mathcal{P}\cap\mathcal{Q}\right)+\mathrm{dim}\left(\mathcal{P}+\mathcal{Q}\right)=2m+1$, то $\mathrm{dim}\mathcal{P}\neq\mathrm{dim}\mathcal{Q}$.
Следовательно,
$$\left[\begin{matrix}
    \left[\begin{matrix}
    \mathrm{dim}\mathcal{P}=\mathrm{dim}\left(\mathcal{P}\cap\mathcal{Q}\right),\\
    \mathrm{dim}\mathcal{Q}=\mathrm{dim}\left(\mathcal{P}+\mathcal{Q}\right);
\end{matrix}\right.\Rightarrow \mathcal{P}\subset\mathcal{Q}\\
\\
    \left[\begin{matrix}
    \mathrm{dim}\mathcal{Q}=\mathrm{dim}\left(\mathcal{P}\cap\mathcal{Q}\right)\\
    \mathrm{dim}\mathcal{P}=\mathrm{dim}\left(\mathcal{P}+\mathcal{Q}\right);
\end{matrix}\right.\Rightarrow \mathcal{Q}\subset\mathcal{P}
\end{matrix}\right.$$
$$\blacksquare$$
\section*{K35.10}
a) $q^{n}$\\
b) $\prod_{i=0}^{n-1}\left(q^{n}-q^{i}\right)$\\
c) $\prod_{i=0}^{n-1}\left(q^{n}-q^{i}\right)$\\
d) $q^{n^2}-\prod_{i=0}^{n-1}\left(q^{n}-q^{i}\right)$\\
e) $\frac{\prod_{i=0}^{k-1}\left(q^{n}-q^{i}\right)}{\prod_{i=0}^{k-1}\left(q^{k}-q^{i}\right)}$\\
f) $q^{n-r}$
\section*{K35.13}
а) Контрпример - три прямые на плоскости.\\
б) $$\forall u\in U\cap(V+W)\hookrightarrow u\in (U\cap V)+(U\cap W)=V+(U\cap W)$$
$$\forall u\in (U\cap V)+(U\cap W)\hookrightarrow u\in V+W, u\in U\Rightarrow u\in U\cap(V+W)$$
\section*{T.3}

\section*{T.4}
Заметим, что существует естественный изоморфизм симметричных матриц с верхнетреугольными (отражение относительно главной диагонали), так что доказанное для симметричных матриц будет верно и для верхнетреугольных.\\
Объединение базисных векторов в пространстве кососимметричных матриц и симметричных матриц (базис описан в \textbf{Б20.6}) образует базис в пространстве $M_{n\times n}(\mathbb{K})$, так что $$M_{n\times n}(\mathbb{K})=U\oplus W=U\oplus W_1$$
$$A_{233}=\begin{pmatrix*}[r]
    1&-1&2\\
    3&-3&6\\
    2&-2&4
\end{pmatrix*}=\begin{pmatrix*}[r]
    0&-3&-2\\
    3&0&2\\
    2&-2&0
\end{pmatrix*}+\begin{pmatrix*}[r]
    1&2&4\\
    0&-3&4\\
    0&0&4
\end{pmatrix*}=\begin{pmatrix*}[r]
    0&-2&0\\
    2&0&4\\
    0&-4&0
\end{pmatrix*}+\begin{pmatrix*}[r]
    1&1&2\\
    1&-3&2\\
    2&2&4
\end{pmatrix*}$$
\section*{Б23.6}5)
$$\varphi(\textbf{x})=\textbf{x}-2(\textbf{x},\textbf{n})\frac{\textbf{n}}{|\textbf{n}|^2}$$
$$\varphi(\lambda\textbf{x})=\lambda\textbf{x}-2(\lambda\textbf{x},\textbf{n})\frac{\textbf{n}}{|\textbf{n}|^2}=\lambda\textbf{x}-2\lambda(\textbf{x},\textbf{n})\frac{\textbf{n}}{|\textbf{n}|^2}=\lambda\varphi(\textbf{x})$$
$$\varphi(\textbf{x}+\textbf{y})=\textbf{x}+\textbf{y}+(\textbf{x}+\textbf{y},\textbf{n})\frac{\textbf{n}}{|\textbf{n}|^2}=\textbf{x}+\textbf{y}+(\textbf{x},\textbf{n})\frac{\textbf{n}}{|\textbf{n}|^2}+(\textbf{y},\textbf{n})\frac{\textbf{n}}{|\textbf{n}|^2}=\varphi(\textbf{x})+\varphi(\textbf{y})$$
$\varphi$ отражает вектор, относительно направления $\textbf{n}$.
\section*{Б23.9}
2)$$x=y=z\qquad\qquad (\tau=(1,1,1))$$
$$\varphi(x)=\frac{(\textbf{x},\tau)}{|\tau|^2}\tau$$
$$E\overset{\varphi}{\mapsto}\frac{1}{3}\mathbb{1}$$
\\
3)$$x+y+z=0\qquad\qquad (\textbf{n}=(1,1,1))$$
$$\psi(x)= \textbf{x}-\frac{(\textbf{x},\textbf{n})}{|\textbf{n}|^2}\textbf{n}$$
$$E\overset{\psi}{\mapsto}E-\frac{1}{3}\mathbb{1}$$
$$\psi=Id_{\mathcal{E}_3}-\varphi$$
\\
\section*{Б23.14}
2)$$\begin{pmatrix*}
    1&0&0\\
    0&0&-1\\
    0&1&0\\
\end{pmatrix*}$$
3)Переходя к ортонормированному базису с ортом вдоль данной прямой:
$$\begin{pmatrix*}
    1/\sqrt{3}&1/\sqrt{6}&1/\sqrt{2}\\
    1/\sqrt{3}&-2/\sqrt{6}&0\\
    1/\sqrt{3}&1/\sqrt{6}&-1/\sqrt{2}\\
\end{pmatrix*}^{-1}\begin{pmatrix*}
    1&0&0\\
    0&-1/2&-\sqrt{3}/2\\
    0&\sqrt{3}/2&-1/2\\
\end{pmatrix*}\begin{pmatrix*}
    1/\sqrt{3}&1/\sqrt{6}&1/\sqrt{2}\\
    1/\sqrt{3}&-2/\sqrt{6}&0\\
    1/\sqrt{3}&1/\sqrt{6}&-1/\sqrt{2}\\
\end{pmatrix*}=$$
$$=\begin{pmatrix*}
    1/\sqrt{3}&1/\sqrt{3}&1/\sqrt{3}\\
    1/\sqrt{6}&-2/\sqrt{6}&1/\sqrt{6}\\
    1/\sqrt{2}&0&-1/\sqrt{2}\\
\end{pmatrix*}\begin{pmatrix*}
    1&0&0\\
    0&-1/2&-\sqrt{3}/2\\
    0&\sqrt{3}/2&-1/2\\
\end{pmatrix*}\begin{pmatrix*}
    1/\sqrt{3}&1/\sqrt{6}&1/\sqrt{2}\\
    1/\sqrt{3}&-2/\sqrt{6}&0\\
    1/\sqrt{3}&1/\sqrt{6}&-1/\sqrt{2}\\
\end{pmatrix*}=$$
$$=\begin{pmatrix*}
    0&1&0\\
    0&0&1\\
    1&0&0
\end{pmatrix*}$$
\section*{Б23.15}$$\mathcal{L}=\mathcal{L}^{\prime}\oplus\mathcal{L}^{\prime\prime}$$1)
$$\mathcal{L}\overset{\varphi}{\to}\mathcal{L}^{\prime}\qquad\qquad\qquad\mathrm{ker}\varphi=\mathcal{L}^{\prime\prime}\qquad\qquad\qquad\mathrm{Im}\varphi=\mathcal{L}^{\prime}$$
\section*{Б23.19}1)
Предположим, что строки матрицы отображения $A_{m\times n}$ линейно зависимы, тогда $\mathrm{rk}(A)<m$. Возьмем такой базис в отображаемом пространстве, что ненулевые строки матрицы линейного отображения $A^{\prime}$ линейно независимы, тогда отображение не будет сюръективным, откуда следует, что $$\mathrm{rk}A=m$$
\section*{Б23.24}
1) Так как $\mathrm{dim}\mathcal{L}=\mathrm{dim}\mathcal{L}^{\prime}$, то матрица $A$ отображения $\varphi$ квадратная. Из доказанного в предыдущем задании следует, что отображение должно быть сюрективным, то есть строчный ранк данной матрицы равен $\mathrm{dim}\mathcal{L}^{\prime}$, и соответственно равен стобцовуму ранку данной матрицы, откуда следует, что данная матрица должна быть невырождена, то есть $$\exists A^{-1}:A^{-1}A\textbf{x}=E\textbf{x}=o\Leftrightarrow \textbf{x}=\textbf{o}$$
2) Из предыдущего пункта следует $\textbf{x}=A^{-1}\textbf{y}$\\
3) Из линейности отображения следует, что $\lambda \textbf{x}$ - решение.$$\varphi(\lambda\textbf{x})=\lambda\varphi(\textbf{x})=\lambda \textbf{y}$$
\section*{Б23.29}5)
$$A=\begin{pmatrix*}
    -2&-2&2\\
    -2&-2&2\\
    -3&-2&2\\
    4&-1&1\\
    6&5&-5\\
\end{pmatrix*}$$
Так как последние два столбца связаны выражением $a_2=-a_3$, то $\mathrm{Im}\varphi=\mathrm{span}(a_1,a_2)$.
$$\mathrm{ker}\varphi=\mathrm{sol}(A\textbf{x}=\textbf{o})$$
$$\begin{pmatrix*}
    -2&-2&2\\
    -2&-2&2\\
    -3&-2&2\\
    4&-1&1\\
    6&5&-5\\
\end{pmatrix*}\rightarrow\begin{pmatrix*}
    -2&-2&2\\
    0&0&0\\
    0&1&-1\\
    0&-5&5\\
    0&-1&1\\
\end{pmatrix*}\rightarrow\begin{pmatrix*}
    1&0&0\\
    0&1&-1\\
    0&0&0\\
    0&0&0\\
    0&0&0\\
\end{pmatrix*}$$
$$\mathrm{ker}\varphi=\mathrm{span}(0,1,1)$$
\section*{Б23.30}1)
Решим систему уравнений:
$$\begin{pmatrix*}[r]
    -1&-5&-4&-3\\
    2&-1&2&-1\\
    5&3&8&1\\
\end{pmatrix*}\begin{pmatrix*}
    x_1\\
    x_2\\
    x_3\\
    x_4
\end{pmatrix*}=\begin{pmatrix*}
    -1\\
    0\\
    1
\end{pmatrix*}$$
$$\begin{pmatrix*}[r]
    -1&-5&-4&-3&\vline&-1\\
    2&-1&2&-1&\vline&0\\
    5&3&8&1&\vline&1\\
\end{pmatrix*}\rightarrow\begin{pmatrix*}[r]
    1&5&4&3&\vline&1\\
    0&11&6&7&\vline&2\\
    0&0&0&0&\vline&0\\
\end{pmatrix*}\rightarrow\begin{pmatrix*}[r]
    11&55&44&33&\vline&11\\
    0&11&6&7&\vline&2\\
    0&0&0&0&\vline&0\\
\end{pmatrix*}\rightarrow$$
$$\rightarrow\begin{pmatrix*}[r]
    11&0&14&-2&\vline&1\\
    0&11&6&7&\vline&2\\
    0&0&0&0&\vline&0\\
\end{pmatrix*}$$
Полный прообраз будет являться линейной оболочной столбцов фундаментальной матрицы $$\begin{pmatrix*}[r]
    14&-2\\
    6&7\\
    11&0\\
    0&11\\
\end{pmatrix*}$$
\section*{Б23.40}1в)
Так как $\left(\frac{t^{m}}{m!}\right)^{\prime}=\frac{t^{m-1}}{(m-1)!}$, то:
$$D=\begin{pmatrix*}
    0&1&0&\cdots&0&0\\
    0&0&1&\cdots&0&0\\
    0&0&0&\ddots&0&0\\
    \vdots&\vdots&\vdots&\ddots&\ddots&\vdots\\
    0&0&0&\cdots&0&1\\
    0&0&0&\cdots&0&0\\
\end{pmatrix*}$$
\section*{Б23.62}3) По формуле перехода в новый базис:
$$A^{'}=S^{-1}AS=\begin{pmatrix*}[r]
    0&-\frac{1}{3}\\
    1&1
\end{pmatrix*}\begin{pmatrix*}[r]
    5&3\\
    -3&-1
\end{pmatrix*}\begin{pmatrix*}[r]
    3&1\\
    -3&0
\end{pmatrix*}=\begin{pmatrix*}[r]
    2&1\\
    0&2
\end{pmatrix*}$$
\section*{Б23.70}1) Перестановка базисных векторов $a_{i}$ и $a_{j}$ соответствует матрице $S=E-E_{ii}-E_{jj}+E_{ij}+E_{ji}$. Нетрудно проветить, что обратна себе. Домножение на неё справа соответствует перестановке строк с теми же номерами.
\section*{T.5}
$$\varphi=\begin{pmatrix*}[r]
    0&-v_{3}&v_{2}\\
    -v_{2}&0&-v_{1}\\
    v_{3}&v_{1}&0\\
\end{pmatrix*}$$
Заметим, что следующее сопоставление изоморфизм (перестановка второй и третьей компоненты первой строки):
$$\begin{pmatrix*}[r]
    0&-v_{3}&v_{2}\\
    -v_{2}&0&-v_{1}\\
    v_{3}&v_{1}&0\\
\end{pmatrix*}\mapsto\begin{pmatrix*}[r]
    0&v_{2}&-v_{3}\\
    -v_{2}&0&-v_{1}\\
    v_{3}&v_{1}&0\\
\end{pmatrix*}$$
Так как сопоставление $\textbf{v}\mapsto\varphi$ - изоморфизм, то
$$\mathbb{R}^{3}\cong M_{3\times3}(\mathbb{R})$$
\section*{T.6*}
a) Заметим, что включение $\mathrm{Ker}\varphi\subset\mathrm{Ker}(\varphi^2)$ выполняется всегда:
$$\forall u\in\mathrm{Ker}\varphi\hookrightarrow\varphi(\varphi(u))=\varphi(\textbf{o})=\textbf{o}$$
Докажем, что при условии $V=\mathrm{Ker}\varphi\oplus\mathrm{Im}\varphi$ выполяется обратное включение:
$$V=\mathrm{Ker}\varphi\oplus\mathrm{Im}\varphi\Leftrightarrow\forall \textbf{v}\in V\ \exists !\ \textbf{u}\in\mathrm{Ker}\varphi,\  \textbf{w}\in\mathrm{Im}\varphi:\textbf{v}=\textbf{u}+\textbf{w}\Rightarrow \varphi(\textbf{v})=\varphi(\textbf{w})\Rightarrow $$
$$\Rightarrow\forall \textbf{t}\in\mathrm{Im}(\varphi)\hookrightarrow\varphi(\textbf{t})=\varphi(\textbf{g})+\varphi(\textbf{h})=\varphi(\textbf{h}):\textbf{g}\in\mathrm{Ker}(\varphi^2),\ \textbf{h}\in\mathrm{Im}(\varphi^2)\Rightarrow$$
$$\Rightarrow\forall \textbf{x}\in\mathrm{Ker}(\varphi^2)\hookrightarrow\varphi(\textbf{x})=\textbf{o}=\varphi(\textbf{h})=\varphi^2(\eta)\Rightarrow \textbf{h}=\textbf{o}=\varphi(\eta)\Rightarrow\eta=\textbf{o}\Rightarrow x\in\mathrm{Ker}\varphi$$
$$\Rightarrow \mathrm{Ker}(\varphi^2)\subset\mathrm{Ker}\varphi\Rightarrow\mathrm{Ker}(\varphi^2)=\mathrm{Ker}\varphi$$
Для эквивалентности $V=\mathrm{Ker}\varphi\oplus\mathrm{Im}\varphi\Leftrightarrow\mathrm{Ker}(\varphi^2)=\mathrm{Ker}$ докажем следствие в обратную другую сторону от противного.\\
Пусть $\mathrm{Ker}(\varphi^2)=\mathrm{Ker}$ и $\mathrm{Ker}\varphi\cap\mathrm{Im}\varphi\neq\mathrm{span}\begin{Bmatrix}
\textbf{o}\end{Bmatrix}$, тогда:
$$\exists\ \textbf{v}\in V:\textbf{v}\in\mathrm{Ker}\varphi,\ \mathrm{Im}\varphi\Rightarrow\exists \textbf{w}\in V:\varphi(\textbf{w})=\textbf{v}\Rightarrow\varphi^2(\textbf{w})=\textbf{o}\Rightarrow w\in\mathrm{Ker}\varphi\Rightarrow\varphi(\textbf{w})=\textbf{o}\Rightarrow \textbf{v}=\textbf{o}\Rightarrow$$
$$\Rightarrow\mathrm{Ker}\varphi\oplus\mathrm{Im}\varphi=V$$
b) Заметим, что $\forall \textbf{v}\in V\hookrightarrow \textbf{v}-\varphi(\textbf{v})=\mathrm{pr}_{\mathrm{Ker}}^{\parallel \mathrm{Im}}\textbf{v}$, действительно:
$$\varphi(\textbf{v}-\varphi(\textbf{v}))=\varphi(\textbf{v})-\varphi^2(\textbf{v})=\textbf{o}$$
Пусть $\textbf{w}=\mathrm{pr}_{\mathrm{Im}}^{\parallel \mathrm{Ker}}\textbf{v}$, тогда:
$$\varphi(\textbf{v})-\varphi^2(\textbf{v})=\textbf{w}-\varphi(\textbf{w})=\textbf{o}$$
Откуда:
$$\varphi(\textbf{w})=\textbf{w}$$
Так как $\textbf{v}=\mathrm{pr}_{\mathrm{Im}}^{\parallel \mathrm{Ker}}\textbf{v}+\mathrm{pr}_{\mathrm{Ker}}^{\parallel \mathrm{Im}}\textbf{v}=\textbf{w}+\textbf{v}-\varphi(\textbf{v})$, то
$$\varphi(\textbf{v})=\textbf{w}$$
То есть $\varphi$ - проектор на $\mathrm{Im}$ вдоль $\mathrm{Ker}$.
\section*{Б23.82*}1) $$\psi\varphi=A_{285}A^{-1}_{289}=\begin{pmatrix*}[r]
    -1&4&-2\\
    2&-1&-1\\
    -1&0&1\\
\end{pmatrix*}$$
\section*{Б23.83}1)$$\varphi=\begin{pmatrix*}[r]
    5&-1\\
    4&1
\end{pmatrix*}$$
$$\varphi^2-6\varphi+9E=(\varphi-3E)^2=O$$
\section*{Б23.71*}1) Данное утверждение сводится к утверждению о том, что любую невырожденную матрицу можно привести к единичной матрице элементарными преобразованиями строк или столбцов - действительно, так как для матрицы линейного отображения работает формула $B=S^{-1}AT$, то получаем желаемое утверждение.
\section*{Б23.74*}4) $$A=\begin{pmatrix*}[r]
    1&-1&2\\
    -1&1&-2\\
    2&-2&4\\
\end{pmatrix*}$$
Приписывая единичную матрицу справа, получаем матрицу перехода к базису отображаемого пространства элементарными преобразованиями строк:
$$\begin{pmatrix*}[r]
    1&-1&2&\vline&1&0&0\\
    -1&1&-2&\vline&0&1&0\\
    2&-2&4&\vline&0&0&1\\
\end{pmatrix*}\rightarrow\begin{pmatrix*}[r]
    1&-1&2&\vline&1&0&0\\
    0&0&0&\vline&1&1&0\\
    0&0&0&\vline&-2&0&1\\
\end{pmatrix*}$$
Приписываем к полученной матрице снизу единичную, получая матрицу перехода в пространстве, содержащем образ отображения.
$$\begin{pmatrix*}[r]
    1&-1&2\\
    0&0&0\\
    0&0&0\\
    \\
    \hline\\
    1&0&0\\
    0&1&0\\
    0&0&1\\
\end{pmatrix*}\rightarrow\begin{pmatrix*}[r]
    1&0&0\\
    0&0&0\\
    0&0&0\\
    \\
    \hline\\
    1&1&-2\\
    0&1&0\\
    0&0&1\\
\end{pmatrix*}$$
Окончательно:
$$A^{\prime}=\begin{pmatrix*}[r]
    1&0&0\\
    1&1&0\\
    -2&0&1\\
\end{pmatrix*}\begin{pmatrix*}[r]
    1&-1&2\\
    -1&1&-2\\
    2&-2&4\\
\end{pmatrix*}\begin{pmatrix*}[r]
    1&1&-2\\
    0&1&0\\
    0&0&1\\
\end{pmatrix*}^{-1}=\begin{pmatrix*}[r]
    1&0&0\\
    0&0&0\\
    0&0&0\\
\end{pmatrix*}$$
\section*{Б23.95*}1)$$\mathrm{rg}\varphi+\mathrm{rg}\psi-m\le\mathrm{rg}(\psi\varphi)\le\min(\mathrm{rg}\varphi,\ \mathrm{rg}\psi)$$
Считая $\mathcal{L},\ \mathcal{M}$ и $\mathcal{N}$ конечномерными, в них можно всегда выбрать базис, тогда второе неравенство очевидно из аналогичного неравенства для матриц.\\
Используя формулу Грассмана преобразуем первое неравенство к:
$$\mathrm{dim}\mathrm{Ker}\varphi+\mathrm{dim}\mathrm{Ker}\psi\ge\mathrm{dim}\mathrm{Ker}(\psi\varphi)$$
Рассмотрим ограничение $\varphi:\mathrm{Ker}(\psi\varphi)\to\mathrm{Ker}\psi$\\
Для образа и ядра данного отображения верны вложения:
$$\varphi\left(\mathrm{Ker}(\psi\varphi)\right)\subset\mathrm{Ker}\psi\qquad\qquad\qquad\mathrm{Ker}\left(\varphi\left(\mathrm{Ker}(\psi\varphi)\right)\right)\subset\mathrm{Ker}\varphi$$
Переходя к размерностям:
$$\mathrm{dim}\mathrm{Ker}(\psi\varphi)=\mathrm{dim}\varphi\left(\mathrm{Ker}(\psi\varphi)\right)+\mathrm{dim}\mathrm{Ker}\left(\varphi\left(\mathrm{Ker}(\psi\varphi)\right)\right)\le\mathrm{dim}\mathrm{Ker}\psi+\mathrm{dim}\mathrm{Ker}\varphi$$
$$\blacksquare$$
\section*{Б12.28}
1)$$\begin{pmatrix*}
    x^{\prime}\\y^{\prime}
\end{pmatrix*}=A\begin{pmatrix*}
    x\\y
\end{pmatrix*}+\begin{pmatrix*}
    c_1\\c_2
\end{pmatrix*}$$
Для неподвижной точки верно:
$$\begin{pmatrix*}
    x^{\prime}\\y^{\prime}
\end{pmatrix*}=\begin{pmatrix*}
    x\\y
\end{pmatrix*}$$
Пусть $A(x_1,y_1)$ и $B(x_2,y_2)$, тогда:$$\begin{pmatrix*}
    x^{\prime}_2-x^{\prime}_1\\y^{\prime}_2-y^{\prime}_1
\end{pmatrix*}=A\begin{pmatrix*}
    x_2-x_1\\y_2-y_1
\end{pmatrix*}=\begin{pmatrix*}
    x_2-x_1\\y_2-y_1
\end{pmatrix*}$$
То есть точка $(x_2-x_1,y_2-y_1)$ тоже неподвижна. Так как уравение прямой $y=kx+b$:$$(x_2-x_1)A\begin{pmatrix*}
    1\\k
\end{pmatrix*}=A\begin{pmatrix*}
    x_2-x_1\\y_2-y_1
\end{pmatrix*}=(x_2-x_1)\begin{pmatrix*}
    1\\k
\end{pmatrix*}$$
Сокращая на $(x_2-x_1)$, как на скаляр, получаем, что данное равенство не зависит от выбора точек на прямой, а значит каждая точка является неподвижной.
$$A\begin{pmatrix*}
    1\\k
\end{pmatrix*}=\begin{pmatrix*}
    1\\k
\end{pmatrix*}$$
2*) Заметим, что инвариантные прямые могут не пересекаться в одной точке только при тождественном преобразовании.
\section*{Б12.40}1)$$\begin{pmatrix*}
    x^{\ast}\\y^{\ast}
\end{pmatrix*}=\begin{pmatrix*}
    a&b\\
    c&d
\end{pmatrix*}\begin{pmatrix*}
    x\\y
\end{pmatrix*}+\begin{pmatrix*}
    e\\f
\end{pmatrix*}$$
Имея это запишем систему уравнений:
$$\left\{\begin{matrix}
    -3=a+e\\
    5=c+f\\
    4=b+e\\
    -3=d+f\\
    0=a+b+e\\
    0=c+d+f
\end{matrix}}\right.\Rightarrow\left\{\begin{matrix}
    a=-4\\
    b=3\\
    c=3\\
    d=-5\\
    e=1\\
    f=2
\end{matrix}}\right.$$
Окончательно:
$$\begin{pmatrix*}
    x^{\ast}\\y^{\ast}
\end{pmatrix*}=\begin{pmatrix*}
    -4&3\\
    3&5
\end{pmatrix*}\begin{pmatrix*}
    x\\y
\end{pmatrix*}+\begin{pmatrix*}
    1\\2
\end{pmatrix*}$$
\section*{Б12.58}8) Из равенства растояний до прямых от центра квадрата находим другие прямые вида $x+2y+a=0$ и $2x-y+b=0$.
$$x+2y=0\qquad\qquad x+2y+6=0\qquad\qquad2x-y+1=0\qquad\qquad2x-y+7=0$$
\section*{Б12.51} Из симметричности гиперболы следует, что начало координат будет неподвижной точкой.
$$\begin{pmatrix*}
    x^{\prime}\\y^{\prime}
\end{pmatrix*}=\begin{pmatrix*}
    a&b\\
    c&d
\end{pmatrix*}\begin{pmatrix*}
    x\\y
\end{pmatrix*}$$
Из $A(5,4)\mapsto B(\sqrt{5},0)$ следует $5c=-4d$ и $5a=\sqrt{5}-4b$.\\
Также:
$$\frac{{x^{\prime}}^{2}}{5}-\frac{{y^{\prime}}^{2}}{4}=\frac{a^2x^2+abxy+b^2y^2}{5}-\frac{c^2x^2+cdxy+d^2y^2}{4}=\frac{x^{2}}{5}-\frac{y^{2}}{4}=1$$
Решая все уравнения вместе, получаем:
$$\begin{pmatrix*}
    x^{\prime}\\y^{\prime}
\end{pmatrix*}=\begin{pmatrix*}
    \sqrt{5}&-\sqrt{5}\\
    -\frac{4}{\sqrt{5}}&\sqrt{5}
\end{pmatrix*}\begin{pmatrix*}
    x\\y
\end{pmatrix*}$$
\section*{Б24.13} Так как характеристичский многочлен данного линейного преобразования будет тоже нечетной степени, то из непрерывности многочленов над $\mathbb{R}$ и разных знаков их пределов при $x\to\pm\infty$ по теореме о промежуточных значениях будет хотя бы один корень, а значит и соответствующий ему собственный вектор.
\section*{Б24.18}2) При отражении $\mathcal{L}$ параллельно $\mathcal{L}^{\prime\prime}$ подпространство $\mathcal{L}^{\prime\prime}$ будет собственным, его базис будет образовывать систему собственных векторов, переходящих в себя, так что $\lambda=1$, каждый вектор из $\mathcal{L}^{\prime}$ будет переходить в обратный по сложению, так что $\lambda=-1$, его базисные векторы будут собственными.
\section*{Б24.20}Используя, что у проекторов собственные значения равны либо 1, либо 0 находим собственные векторы и соответствующие им собственные подпространства.\\
2) Собственными подпространствами будут данная прямая и плоскость перпендикулярная ей.\\
3) Собственными подпространствами будут данная плоскость и прямая перпендикулярная ей.
\section*{Б24.22*}
1) Ранг матрицы $A$ не превосходит 1, так как она является произведением  матриц ранга 1. \\
Заметим, что $a=(a_1,\cdots,a_n)^{T}$ - собственный вектор данного преобразования.
$$Aa=aba=(ba)a=\lambda a=a\mathrm{tr}A$$
Остальные собственные векторы будут соответствовать собственному значению 0, и будут являться столбцами фундаментальной матрицы системы $A\xi=o$. Как изветстно, их будет $n-\mathrm{rk}A=n-1$\\
2)В прошлом задании было определено, что собственному значению 0 соответствует $n-1$ векторов, что равно размерности соответствующего корневого подпространтсва. Если $\mathrm{tr}A=0$, то корневое подпространство будет той же размерности, в однако кратность значения 0 будет $n$, поэтому при $\mathrm{tr}A=0$ матрица не диагонализируема. (используя это, следующий пункт решается в уме)
\section*{Б24.30}
7)$$A=\begin{pmatrix*}
    1&1\\
    1&1\\
\end{pmatrix*}$$
$$\chi_A(\lambda)=\lambda^2-2\lambda\Rightarrow\left[\begin{matrix}
    \lambda=0\\
    \lambda=2
\end{matrix}\right.$$
Соответствующие собственные векторы $a_1=(1,1)^{T}$ и $a_2=(-1,1)^{T}$
$$A^{\prime}=\frac{1}{2}\begin{pmatrix*}
    1&1\\
    -1&1\\
\end{pmatrix*}\begin{pmatrix*}
    1&1\\
    1&1\\
\end{pmatrix*}\begin{pmatrix*}
    1&-1\\
    1&1\\
\end{pmatrix*}=\begin{pmatrix*}
    2&0\\
    0&0\\
\end{pmatrix*}$$
Данное преобразование соответствует композиции проектора на прямую $y=x$ и удлинения вдоль неё вдвое.\\
19)$$A=\begin{pmatrix*}
    4&1&-1\\
    2&5&-2\\
    4&4&-1\\
\end{pmatrix*}$$
$$\chi_A(\lambda)=\lambda^3+8\lambda^2-21\lambda+18\Rightarrow\left[\begin{matrix}
    \lambda=3\\
    \lambda=3\\
    \lambda=2
\end{matrix}\right.$$
Соответствующие собственные векторы $a_1=(-1,1,0)^{T}$, $a_2=(1,0,1)^{T}$ и $a_3=(1,2,4)^{T}$
$$A^{\prime}=-\begin{pmatrix*}
    2&3&-2\\
    4&4&-3\\
    1&1&-1
\end{pmatrix*}\begin{pmatrix*}
    4&1&-1\\
    2&5&-2\\
    4&4&-1
\end{pmatrix*}\begin{pmatrix*}
    -1&1&1\\
    1&0&2\\
    0&1&4
\end{pmatrix*}=\begin{pmatrix*}
    3&0&0\\
    0&3&0\\
    0&0&2
\end{pmatrix*}$$
30)$$A=\begin{pmatrix*}
    2&5&1\\
    -1&-3&0\\
    -2&-3&-2\\
\end{pmatrix*}$$
$$\chi_A(\lambda)=-\lambda^3-3\lambda^2-3\lambda-1\Rightarrow\lambda=-1$$
Соответствующий собственный вектор $a=(-2,1,1)^{T}$
\section*{Б24.37*}Общим методом для всей задачи будет проверка размерности подпространства $\mathrm{Ker}(A-E)^2$ - для одной матрицы оно будет 2, почему можно определить подобие.\\
3) Подобие сохраняет инвариантные функции, так что матрица со следом $0$ подобна $\mathrm{diag}(1,-1,0)$.
\section*{Б24.42}1) В задаче \textbf{Б23.40} описан базис, в котором матрица оператора дифференцирования принимает простейший вид. Откуда:
$$\chi_A(\lambda)=(-\lambda)^n$$
Так, собственное значение равно 0. $A^n=O$, так что собственный вектор равен $(0,\cdots,0,1)^{T}$. Существование единственного собственного вектора согласуется с тем, что матрица оператора дифференцирования в указанном базисе является жордановой клеткой $J(0)$
\section*{Б24.53} Как известно, симметричные и кососимметричные матрицы образуют прямую сумму, в их базисе транспонирование будет соответствовать такому отображению:
$$E_{+}+E_{-}\mapsto E_{+}-E_{-}$$
Другими словами, транспонирование - отражение матрицы вдоль антисимметричных матриц вдоль симметричных
\section*{Б24.55}1) Для преобразования пространства матриц порядка 2, заданного формулой $\varphi(X)=\begin{pmatrix*}
    -4&0\\
    1&4
\end{pmatrix*}X$, матрица преобразования равна
$$P=\begin{pmatrix*}
    -4&0&0&0\\
    0&-4&0&0\\
    1&0&4&0\\
    0&1&0&4
\end{pmatrix*}$$
Собственные значения равны $\pm4$, им соответствуют собсвенные векторы
$$\begin{pmatrix*}[r]
    8&0\\
    -1&0
\end{pmatrix*}\quad\quad\quad\quad\begin{pmatrix*}[r]
    0&8\\
    0&-1
\end{pmatrix*}\quad\quad\quad\quad\begin{pmatrix*}
    0&0\\
    8&0
\end{pmatrix*}\quad\quad\quad\quad\begin{pmatrix*}
    0&0\\
    0&8
\end{pmatrix*}$$
В их базисе матрца преобразования равна
$$P^{\prime}=\begin{pmatrix*}
    -4&0&0&0\\
    0&-4&0&0\\
    0&0&4&0\\
    0&0&0&4
\end{pmatrix*}$$
\section*{Б23.98*}2)$$\varphi^2=E$$
$$\varphi^2-E=\varphi^2-E^2=(\varphi-E)(\varphi+E)=O$$
Заметим, что $(\varphi-E)\xi=(\varphi+E)\xi\Leftrightarrow\xi=o$, так что $\mathrm{Im}(\varphi-E)\oplus\mathrm{Im}(\varphi+E)=V$
\section*{K40.11*}
\section*{K40.12*}a)
\section*{T.7}Характеристический многочлен матрицы $A=\begin{pmatrix*}
    0&1&0\\
    0&0&1\\
    1&2&0
\end{pmatrix*}$ равен
$$\chi_{A}(\lambda)=-\lambda^3+2\lambda+1$$
Методом подстановки находим, что над полем $\mathbb{F}_{3}$ существует только одно собственное значение 1, а над полем $\mathbb{F}_{5}$ - 3 и 4
\section*{T.8} Так как $\mathrm{tr}A=1+2\cos\alpha$, то $\alpha=\arccos\left(\frac{\mathrm{tr}A-1}{2}\right)$
\section*{T.9} Матрицы $O$ и оператора дифференцирования имеют однинаковые характеристические многочлены, но не являются подобными. Для диагонализируемых матриц данное утверждение верно, так как данные матрицы будут обе подобны соответствуюшей диагональной матрице, одинаковой для них.
\section*{T.10*}
\section*{Б24.70}Предположим обратное - пусть $\exists v\in U\subset V:\varphi(v)\not\in U$, тогда $\varphi(v)\not\in \mathrm{Im}\varphi$, однако $\varphi(v)\in\mathrm{Im}\varphi$. Противоречие. 
\section*{Б24.68}4) Из комутативности многочленов:
$$\forall v:p(\varphi)(v)=o\Leftrightarrow\varphi(p(\varphi)(v))=0\Rightarrow(p(\varphi)(\varphi (v)))=0$$
$$\blacksquare$$
\section*{Б24.71*}Для любого вектора, лежащего в инвариантном подпространстве, как образ будет, так и прообраз будут лежать в этом пространстве, поэтому инволюция и обратное к ней преобразование будут иметь одни и те же инвариантные подпространства.
\section*{Б24.74}
1) Следует из свойств дополненых миноров.\\
2*)
\section*{Б24.77} Прямая вдоль $\textbf{a}$ и плоскость, перпендикулярная ей.
\section*{Б24.78*} $2^{n}+1$ - подпространства содержащие различные собственные векторы, а также нулевое подпространство
\section*{T.11} Если матрица не нильпотентна, то к нулевому подпространству добавляется ещё одно, направленное вдоль собственного вектора.
\section*{T.12*}
a)
b)
c)
\section*{Б24.126}5)$$A=\begin{pmatrix*}[r]
    1&2&1\\
    1&2&4\\
    -1&-2&-3
\end{pmatrix*}\quad\quad\quad\quad A^2=\begin{pmatrix*}[r]
    2&4&6\\
    -1&-2&-3\\
    0&0&0
\end{pmatrix*}\quad\quad\quad\quad A^{3}=O$$
\section*{Б24.127}
7)$$A=\begin{pmatrix*}[r]
    2&-5&-4\\
    -3&16&12\\
    4&-20&15
\end{pmatrix*}$$
$$\chi_A{\lambda}=(\lambda-1)^3$$
$$B=\begin{pmatrix*}[r]
    1&-5&-4\\
    -3&15&12\\
    4&-20&-16
\end{pmatrix*}$$
Собственные векторы, соответствующие значению 1, равны $(5,1,0)^{T}$ и $(4,0,1)^{T}$. Линейная оболочка столбцов $B$ с собственным подпространством пересекается по $(1,-3,4)^{T}$, так что присоединенный вектор $(1,0,0)^{T}$ (вообще говоря, так как $\mathrm{Rk}B=1$, то пересечение будет по любому столбцу $B$).
$$J=\begin{pmatrix*}
    5&4&1\\
    1&0&0\\
    0&1&0
\end{pmatrix*}\qquad\qquad\qquad A^{\prime}=JAJ^{-1}=\begin{pmatrix*}
    1&0&0\\
    0&1&1\\
    0&0&1
\end{pmatrix*}$$
Из жордановой формы заключаем, что $\mu_A(\lambda)=(\lambda-1)^2$\\
15)$$A=\begin{pmatrix*}[r]
    1&1&-1&1\\
    1&-1&1&-1\\
    -3&-1&1&-1\\
    -3&1&-1&1
\end{pmatrix*}$$
$$\chi_A{\lambda}=\lambda^3(\lambda-2)$$
Собственные векторы, соответствующие значению 0, равны $(0,1,1,0)^{T}$ и $(0,-1,0,1)^{T}$. Линейная оболочка столбцов $A$ с собственным подпространством пересекается по $(0,-2,2,4)^{T}$, так что присоединенный вектор $(-1,1,0,0)^{T}$. Cобственный вектор соответствующий значению 2 равен $(-1,0,4,1)^{T}$
$$J=\begin{pmatrix*}[r]
    -1&0&0&-1\\
    0&1&-1&1\\
    4&1&0&0\\
    1&0&1&0
\end{pmatrix*}\qquad\qquad\qquad A^{\prime}=JAJ^{-1}=\begin{pmatrix*}
    2&0&0&0\\
    0&0&0&0\\
    0&0&0&1\\
    0&0&0&0
\end{pmatrix*}=\mathrm{diag}(2,0,J_2(0))$$
Из жордановой формы заключаем, что $\mu_A(\lambda)=(\lambda-2)\lambda^2$\\
18)$$A=\begin{pmatrix*}[r]
    0&0&1&0\\
    0&0&0&1\\
    3&4&0&0\\
    -1&-1&0&0
\end{pmatrix*}$$
$$\chi_A{\lambda}=(\lambda^2-1)^2$$
Собственный вектор, соответствующие значению 1, равен $(0,1,-2,1)^{T}$, присоединенный к нему - $(0,1,-2,1)^{T}$.\\
Собственный вектор, соответствующие значению -1, равен $(0,1,-2,1)^{T}$, присоединенный к нему - $(0,1,-2,1)^{T}$.
$$J=\begin{pmatrix*}[r]
    -1&0&0&-1\\
    0&1&-1&1\\
    4&1&0&0\\
    1&0&1&0
\end{pmatrix*}\qquad\qquad\qquad A^{\prime}=JAJ^{-1}=\begin{pmatrix*}[r]
    1&1&0&0\\
    0&1&0&0\\
    0&0&-1&1\\
    0&0&0&-1
\end{pmatrix*}=\mathrm{diag}(J_2(1),J_2(-1))$$
Из жордановой формы заключаем, что $\mu_A(\lambda)=\lambda^2-1$\\
\section*{Б24.141}1) Собственное подпространство одного преобразования будет инвариантным для другого, так как характеристический многочлен ограничения на данном подпространстве будет иметь корень в поле комплекных чисел, то они имеют общий собственный вектор.
\section*{Б24.26}
Исходя из того, что характеристические числа различны, заключаем, что матрица диагонализируема.\\
1)$\lambda_i^2$\\
4*)$p(\lambda_i)$
\section*{K41.8} От противного. $\blacksquare$
\section*{K41.17*}
\section*{K41.18}$$A\sim J_A\sim J_A^{T}\sim A^{T}$$
\section*{K41.30*} $$A^3-A^2=A^2(A-E)=O$$
Минимальным многочленом будет $\mu_A(\lambda)=\lambda^2(\lambda-1)$, поэтому в блочно-диагональном виде матрицы могут присутствовать только $1$, $0$ и $J_2(0)$
\section*{K41.15*}
\section*{T.13}
\section*{T.14*}
Так как характеристический многочлен является аннулирующим для оператора, то $$\varphi^{-1}\chi(\varphi)=O$$
\section*{T.15*} Слудует из принципа Дирихле и $\textbf{T.17(c)}$
\section*{T.16*}
\section*{T.17*}
a) Многочлены от матриц комутативны, так как любая матрица перестановочна с собой\\
b) $L(A)\cong \mathbb{R}\left[\lambda\right]/\mu_A(\lambda)=\mathbb{R}\left[\lambda\right]/(\lambda^2+1)$\\
c) Так как минимальный многочлен является аннулирующим, то $$\forall f(A)\hookrightarrow f(A)=g(A)\mu_A(A)+r(A)=r(A):\mathrm{deg}\ r(A)<\mathrm{deg}\ \mu_A(A)=s$$
Так что $\mathrm{deg}\ r(A)\le s-1$, откуда размерность $\mathrm{dim}L(A)=s=\mathrm{deg}\ \mu_A(A)$\\
d)
e)
f)
\section*{T.18} Матрица оператора состоит из $\begin{Bmatrix}
    J_1(0),J_2(0),J_1(1),J_2(1)
\end{Bmatrix}$
\section*{T.19} Так как характеристический многочлен оператора дифференцирования равен $\chi_A(\lambda)=\lambda^9$, то собственные числа равны только 0. Собственным подпространством получается пространство многочленов не выше степени 2.
$$A=\mathrm{diag}(J_{7}(0),J_{1}(0),J_{1}(0))$$
Минимальным многочленом будет $\lambda^7$
\section*{T.20}
a) $$\begin{pmatrix*}
    x_{n+1}\\
    x_{n}
\end{pmatrix*}=\begin{pmatrix*}
    4&-4\\
    1&0
\end{pmatrix*}\begin{pmatrix*}
    x_{n}\\
    x_{n-1}
\end{pmatrix*}$$
Продолжая запись:
$$\begin{pmatrix*}
    x_{n+1}\\
    x_{n}
\end{pmatrix*}=\begin{pmatrix*}
    4&-4\\
    1&0
\end{pmatrix*}^{n-2}\begin{pmatrix*}
    x_{2}\\
    x_{1}
\end{pmatrix*}$$
Найдем жорданов базис:
$$\chi_A(\lambda)=(\lambda-2)^2$$
Соответственно, $(2,1)^{T}$ - собственный, $x_n=2^{n-1}x_1$.
$$\begin{pmatrix*}
    2&1\\
    0&2
\end{pmatrix*}^{n}=\begin{pmatrix*}
    2^n&n2^{n-1}\\
    0&2^n
\end{pmatrix*}$$
$$\begin{pmatrix*}
    2&-4\\
    1&-2
\end{pmatrix*}\textbf{x}=\begin{pmatrix*}
    2\\
    1
\end{pmatrix*}$$
$$x=\begin{pmatrix*}
    1\\0
\end{pmatrix*}\Rightarrow J=\begin{pmatrix*}
    2&1\\
    1&0
\end{pmatrix*}$$
$$\begin{pmatrix*}
    x_{n+1}\\
    x_{n}
\end{pmatrix*}=J\begin{pmatrix*}
    4&-4\\
    1&0
\end{pmatrix*}^{n-2}J^{-1}\begin{pmatrix*}
    x_{2}\\
    x_{1}
\end{pmatrix*}=\begin{pmatrix*}
    (n+1)2^{n-2}&-n2^{n-1}\\
    n2^{n-3}&(1-n)2^{n-2}
\end{pmatrix*}\begin{pmatrix*}
    x_{2}\\
    x_{1}
\end{pmatrix*}$$
$$x_n=(n+1)2^{n-2}x_2-n2^{n-1}x_1=n2^{n-2}(x_2-2x_1)+2^{n-2}x_2$$
b)$$x_n=n2^{n-2}(x_2-2x_1)+2^{n-2}x_2=(n-1)2^{n}$$
\section*{T.21*}
\end{document}
